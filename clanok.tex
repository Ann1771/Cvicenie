% Metódy inžinierskej práce

\documentclass[10pt,twoside,english,a4paper]{article}

\usepackage[slovak]{babel}
%\usepackage[T1]{fontenc}
\usepackage[IL2]{fontenc} % lepšia sadzba písmena Ľ než v T1
\usepackage[utf8]{inputenc}
\usepackage{graphicx}
\usepackage{url} % príkaz \url na formátovanie URL
\usepackage{hyperref} % odkazy v texte budú aktívne (pri niektorých triedach dokumentov spôsobuje posun textu)

\usepackage{cite}
%\usepackage{times}

\pagestyle{headings}

\title{Using of seriouse game for learning languages in context of Irish language\thanks{Semestrálny projekt v predmete Metódy inžinierskej práce, ak. rok 2022/23, vedenie: ==Anna Pylypenko}} % meno a priezvisko vyučujúceho na cvičeniach

\author{Anna Pylypenko\\[2pt]
	{\small Slovenská technická univerzita v Bratislave}\\
	{\small Fakulta informatiky a informačných technológií}\\
	{\small \texttt{xpylypenko@stuba.sk}}
	}




\date{\small 20. september 2022} % upravte



\begin{document}

\maketitle

\begin{abstract}
The idea of learning languages using not only textbooks and notebooks has long been with people. Digital game-based language learning (DGBLL) has become an increasingly popular topic in the field of digital educational games. DGBLL can provide learners with an enjoyable gaming experience as well as enhancing their language learning experience. The need for engaging approaches to the teaching and learning of minority or endangered languages has also led to greater interest in the application of DGBLL approaches. In this article I will be reviewing a game for learning Irish. As this language is an endangered one, the article aims to predict the effectiveness of this method of learning. The article will cover the topic of using games as the main method of language learning, creating a language environment through games, keeping students interested, advantages and disadvantages\\
Keywords: Second Language Acquisition, games and learning, language learning, serious games, Digital Educational Games\\
\ldots

\end{abstract}


\section{Introduction}

Irish is the official first language of the Irish state, however at present it is only spoken as a community language by 3% of the population (Central Statistics Office, Ireland,
2012). It is a compulsory subject in Irish schools, with daily language classes for the vast majority of children between the ages of 5-18. Despite this significant investment
of time and resources, almost one in three teenagers claimed to be unable to speak the language in the 2011 census, and research in the primary school has shown a sharp decline in standards of attainment since the 1980s (Harris, Forde, Archer, Nic Fhearaile, \& O’Gorman, 2006). \\
For most primary-age children, their only contact with the Irish language is in the daily Irish lesson, a few classroom phrases in school and perhaps incidental Irish use outside of the school context, for example in place names and road signage. While Irish children and adults tend to be postiviely disposed towards the language (McCoy, Smyth,  \& Fitzpatrick, 2012; MORI Ireland, 2005), motivation can be an issue for children who limited opportunity to use Irish outside of school in an authentic language community (Ó Laoire, 2005). A recent study by Devitt et al highlighted the problem of primary school children’s excess disengagement with Irish when compared with school in general and with Maths and English, and suggested a link between this disengagement and a lack of exposure to Irish outside of school (Devitt, Condon, Dalton, O’Connell, \& Ní Dhuinn, in press). Technology may hold the key to connecting Irish speakers together to form a virtual language community and to create an environment where Irish is used to communicate in meaningful and authentic ways. Uveďte explicitne štruktúru článku. Tu je nejaký príklad.\\\\
Základný problém, ktorý bol naznačený v úvode, je podrobnejšie vysvetlený v časti~\ref{Advatages}.
Dôležité súvislosti sú uvedené v častiach~\ref{Advatage} a~\ref{Advatage}.
Záverečné poznámky prináša časť~\ref{zaver}.



\section{Advantages} \label{Advantages}
\subsection{Motivation}
Success and happiness will  arise in an atmosphere  where there is joy and  there are not any losers.  Success and happiness will  inspire student motivation. “Motivation has been widely accepted as one of the key factors that influence success in second/foreign language  (L2)  learning”  (C.  Ng \&  P.  Ng,  2015). Motivation  is the  desire  to  learn. Paris  and  Yussof  (2012) asserted  learners’ enthusiasm towards  games as  an internal  drive which  helps students  practice both  form and  function. Similarly,  Ersöz (2000) expressed “games  are highly motivating  since they are  amusing and at  the same time  challenging. Furthermore, they  employ meaningful and useful language in real contexts”.  According to Ersöz games are motivating because of the fun factor. Motivation that comes with game  playing is highly  valued in language teaching,  because it reduces  language learning  anxiety. Anxiety is  a serious affective  filter that  gets in  the way of  learning. In  this respect,  Steinberg (1986)  added that “playing  games takes the drudgery out  of learning  and, thus, provides  motivation” (p. x).  Upon creating a  happy class  atmosphere, learners will  have a positive attitude towards learning a language. In such a pleasant learning environment the stress factor will be limited. Teaching language in an enjoyable atmosphere through games will enable continuity of learning activities which will lead to more language exposure. This means an increased  language input for learners which  is fundamental in both language learning and acquisition (Çakır, 2004, p.103).  In a qualitative research  (Köksal, Çekiç \& Beyhan,  2014), students expressed  their motivation  raised their enthusiasm towards English classes positively and enhanced their inner strength (p. 87). Learners feel relaxed through games and they are not depressed  as there will not  be any fear of failure.  Using games in  teaching second language  also lowers  learners’ affective filter. Çakır (2004) claimed “creating a psychologically secure setting in the classroom enhances learning no matter how difficult the subject or what the level of student are” (p. 104). In the interview by Berkant \& Arslan Avşar (2015) students of the experimental group expressed they started to love EFL classes thanks to the games and they also added that games helped them remember learned topics in the exams (p.189). These points highlight the positive impact of games on motivation in a language classroom. 
 \subsection{Participation in Class }
In every class there are reluctant and shy students which makes teachers have difficulty teaching them. Teachers can observe how games change the participation of such learners because games reduce timidity. Young learners are eager to answer the questions in games even if they are not normally (Çakır, 2004). Steinberg (1986) also explained games make shy or linguistically weak students participate especially when the object is just to have fun, not to win. Moreover, an active language learner will get the linguistic affordances [mutual relationship between an organism and the feature around it] which help them to enrich their experience of language practice and use them for linguistic action (van Lier, 2000: 252). There are precautions to consider for teachers when they call on participation. Teachers must be wise about correcting mistakes during games in class. A teacher should not interfere too often to correct errors. Some mistakes can be negligible so as not to discourage participation. The aim of the activity is to use the language, not to be fixated on the rules. So, if the correction is less frequent, participation will be more commonplace. According to Wright et  al.  “The greatest  mistake (if oral ability  is aim) is  for the learner not  to speak  at all” (2006, p.3). Especially young learners are good at learning unconsciously. If they feel that they are having fun, they will be eager participants. Erdoğan  (2015) suggested in the interview with young learners in his research even the unwilling learners take part in the language activities and feel positive attitudes towards the class while participating in language games. This is because language games change the negative atmosphere of the class (p. 98). Furthermore, they will start using the language automatically. Tyson (1998) also emphasized “a language game should keep all of the students involved and interested” (para. 1, www.english.daejin.ac.kr). Once the language is used, the language point will be learned in real practice and thus, more frequent participation means higher  levels of language performance. 


\begin{figure*}[tbh]
\centering
%\includegraphics[scale=1.0]{diagram.pdf}
Aj text môže byť prezentovaný ako obrázok. Stane sa z neho označný plávajúci objekt. Po vytvorení diagramu zrušte znak \texttt{\%} pred príkazom \verb|\includegraphics| označte tento riadok ako komentár (tiež pomocou znaku \texttt{\%}).
\caption{Rozhodujúci argument.}
\label{f:rozhod}
\end{figure*}



\section{Why is game better way to learn language?} \label{ina}
Learners tend to use their  L1 a lot when they have fewer opportunities  with  the target language. As Tomlinson and Masuhara (2009)  stated  “learners  need  opportunities  to  use language  to  try  to  achieve  communicative  outcomes”  (p.10).  They  need reasonable practice  in class; pointing  to a pencil  and asking what  it is,  is totally unreasonable  and this creates  a dull learning environment. “Language and activity is an authentic combination for these learners [young learners] – it is one they use in L1” (Roth,  2007,  p.26).  Authenticity  brings  interest  and  enthusiasm  into  the  language  classroom.  
Language in games is meaningful and contextualized. Also, we can repeat games several times in various ways and the language used in games is semantically focused and meaningful to learners (Tomlinson and Masuhara,  2009). Games  also combine form  and function of grammar  in TL  (target language). This  means communicative aspects of the language can be achieved by emphasizing both fluency and accuracy. Students gain both accuracy and fluency since real life like materials embedded in games contain both of these elements (Köksal, Çekiç \& Beyhan,  2014, p. 83). Games enable learners and teachers to go beyond traditional methods and use the learned language point in real life context. Tyson (1998) said “a successful game should encourage students to focus on the use of language rather than the language itself” (para. 1). Language games are ideal tools to make learners use the language. They produce new utterances instead of memorization of phrases. They do not just memorize how to fill in the blanks or write a sentence in simple present tense, but also communicate in the target language. This is because they want to be active in the learning process. Wright et al. (2006) stated “games provide one way of helping the learners to experience the language rather than merely study it” (p. 2). When a teacher brings old fashioned language practices to the class such as filling in the blanks with the correct word form, language becomes artificial. These outdated practices do not inspire students and make them memorize the language items out of context. Learners cannot figure out how to use this memorized language in real life situations.  Games provide meaningful and  real  use of language. Roth (2007) also claimed that: ‘‘Playing is a child’s natural way of learning. A game with all its rules and interaction is a mini social world in which children prepare themselves, little  by little,  to enter society.  Games also develop  the child’s automatic  use of  a foreign  language, coordination, cognitive thought, etc.’’ (p. 26). Just like a child learns step by step to act in society, a learner also practices its position in the second language in a pretend situation. Thus, games increase the meaning of language because it is the participation of language which makes us understand why we learn what we learn\\

\section{Game aspects }
The game world is a magical one in which ancient evil spirits are attempting to deny access to the ancient mythological tales by placing them under a spell, to cause people to lose their memory of their past. The player's challenge is to decipher these spells in order to restore the tales before they are sealed and lost forever. There are many different spells (ciphers) and stages before all the evil spells can be lifted and the story is restored. Figure 1 Game Interface Players accumulate points when they correctly identify ciphered words and lose points when they fail to spot a ciphered word or incorrectly identify Faoi Gheasa: an adaptive game for Irish language learning Liang Xu1 , Elaine Uí Dhonnchadha2 and Monica Ward1 1Dublin City University, Ireland 2Trinity College Dublin, Ireland 133  a ciphered word. Players can use their points to buy hints if they wish, which means that players with a minimal amount of Irish can enjoy playing the game. If a player cannot find all of the ciphered words on a page, they are given the choice to 'change the ending' by writing some text in Irish, or to abandon the attempt in which case they will be presented with the same page but with easier ciphers. The game is developed using Unity (client) and Photon (server).\\

\section{L1-L1 issues}
 Irish has a complex role in Irish society. While not all members of society value the language for cultural and heritage reasons, for many Irish citizens and the Irish diaspora around the world, the Irish language has great cultural significance and they have a strong desire to acquire and improve their Irish language skills, and to ensure that their children are confident users of the language. In learning a second language (L2), features which are not present in their first language (L1) often 134  present additional challenges for the learner (Laufer \& Eliasson, 1993; Schepens, Van Hout, \& Van der Slik, 2022; Vainio, Pajunen, \& Hyönä, 2014). The majority of L2 learners of Irish have English as their L1. There are many linguistic differences between Irish and English, all of which can create barriers to the learning of Irish, a minority language in the shadow of English. One difficulty for L1 English language speakers learning Irish is that orthography system is different from English yet uses the same Latin alphabet. While the Irish orthography system is opaque, it is more regular than English. However, the rules of the orthography system are not generally taught to students, and they are often left to decipher it themselves. Often, students do not see the patterns, and this hampers their learning. They automatically ‘map’ the English soundorthography system to Irish, which is not always a successful approach. For example, the word teach meaning 'house' in Irish is pronounced quite differently from the word 'teach' in English. Another difficulty for Irish language learners is that Irish has a complex system of initial mutations. This is a defining feature of the Celtic languages, which affects the initial phonemes of verbs, nouns, pronouns, adjectives, and some functional categories. The initial mutations on nouns, (and the word classes which modify and agree with a head noun), vary according to the gender of the noun i.e., whether the noun is masculine or feminine. At the level of morphology, Irish verbs are inflected for tense/mood, person and number, and nouns are inflected for number and case, the formation of which varies according to the gender of the noun. Features of Irish such as initial mutation, gender agreement, and case marking will be unfamiliar to learners whose first language is English. Often Irish language learners are oblivious to the morphological and grammatical information encoded in a word and therefore lose vital clues when trying to understand written and spoken language. For example, in (1) Bhí 'was' has an initial mutation for past tense, mhór 'big' has an initial mutation to signal agreement with a feminine noun tine 'fire', mbradán 'salmon' has initial mutation as it is the object of the preposition and definite article faoin 'under the', and feasa 'knowledge' is in the genitive case to signify its relationship to mbradán 'salmon'. (1) Bhí tine mhór faoin mbradán feasa. Was fire big under.the salmon knowledge. 'There was a big fire under the salmon of knowledge' In this game we encourage noticing of spelling orthography by introducing cipher errors into the stories. Most cipher errors are not errors which a learner would naturally make e.g., swapping the first half of a word with the second half, doubling the last letter, or removing all vowels. These types of errors encourage noticing, are relatively easy to spot, and minimise the risk of familiarising the learners with misspellings. In Figure 2 we have an example of the "Double Tail" cipher which doubles the last letter of a word, e.g. Is 'is' has become Iss and mé 'me' has become méé. Figure 2 Example of a cipher and noun gender colour coding In this experiment we encourage the noticing of noun gender which is a central feature of the morpho-syntax of Irish. English language speakers are generally unfamiliar with this grammatical feature of Irish. We do this as part of the game narrative by presenting nouns in distinct colours depending on their gender. In this way we facilitate the noticing of the two distinct types of noun. Some of the more complex ciphers remove the colour coding from nouns, and certain ciphers affect nouns of one gender or the other. Therefore noticing and remembering that individual words are affiliated to either the Water Spirit (blue, masculine nouns) or the Fire Spirit (red, feminine nouns) is an advantage in later stages of the game. In Figure 2 we see that marúch 'mermaid' is red and dúlachán 'dark one' is blue.


Základným problémom je teda\ldots{} Najprv sa pozrieme na nejaké vysvetlenie (časť~\ref{ina:nejake}), a potom na ešte nejaké (časť~\ref{ina:nejake}).\footnote{Niekedy môžete potrebovať aj poznámku pod čiarou.}

Môže sa zdať, že problém vlastne nejestvuje\cite{Coplien:MPD}, ale bolo dokázané, že to tak nie je~\cite{Czarnecki:Staged, Czarnecki:Progress}. Napriek tomu, aj dnes na webe narazíme na všelijaké pochybné názory\cite{PLP-Framework}. Dôležité veci možno \emph{zdôrazniť kurzívou}.
\subsection{Nejaké vysvetlenie} \label{ina:nejake}

Niekedy treba uviesť zoznam:

\begin{itemize}
\item jedna vec
\item druhá vec
	\begin{itemize}
	\item x
	\item y
	\end{itemize}
\end{itemize}

Ten istý zoznam, len číslovaný:

\begin{enumerate}
\item jedna vec
\item druhá vec
	\begin{enumerate}
	\item x
	\item y
	\end{enumerate}
\end{enumerate}


\subsection{Ešte nejaké vysvetlenie} \label{ina:este}

\paragraph{Veľmi dôležitá poznámka.}
Niekedy je potrebné nadpisom označiť odsek. Text pokračuje hneď za nadpisom.



\section{Dôležitá časť} \label{dolezita}


\section*{My introduction}
Hello, my name is Ann/ I am from Ukraine.
\section*{My hobbies}




I love drawing, but also i love sport and reading.
I don`t like cabbage. I  really enjoy having fun with friends and watching movies. However, I am too nervous all of the time. i am very afrayed of missing and making  abig mistakes.
\subsection{My favorite food}
mango
avocado
banana

as you see i like fruits
\section{Ešte dôležitejšia časť} \label{dolezitejsia}


\section{Nový članok blablabla}
\subsection{ppprepfjv}



\section{Záver} \label{zaver} % prípadne iný variant názvu



%\acknowledgement{Ak niekomu chcete poďakovať\ldots}


% týmto sa generuje zoznam literatúry z obsahu súboru literatura.bib podľa toho, na čo sa v článku odkazujete
\bibliography{literatura}
\bibliographystyle{abbrv} % prípadne alpha, abbrv alebo hociktorý iný
\end{document}
